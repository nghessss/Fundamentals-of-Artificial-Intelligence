\section{Output Evaluation}
All the test cases is run with max iterations of 100. Increasing this value may lead to the different output or run time, however, with the small number of clusters, the growth of iteration does not affect the algorithm performance due to the early stopping
\pagebreak
\subsection{Result}
\subsubsection{lena image}
\quad This image's size is 264KB

\paragraph{Random Mode}
\quad 
\begin{figure}[htbp]
    \centering
    \begin{minipage}{0.45\textwidth}
        \centering
        \includegraphics[scale=0.4]{lena/original.png}
        \caption{Original}
    \end{minipage}\hfill
    \begin{minipage}{0.45\textwidth}
        \centering
        \includegraphics[scale=0.4]{lena/3_random.png}
        \caption{$k\_clusters = 3$, $mode = 'random'$}
    \end{minipage}
    
    \vspace{0.5cm} % Adjust vertical spacing between rows
    
    \begin{minipage}{0.45\textwidth}
        \centering
        \includegraphics[scale=0.4]{lena/5_random.png}
        \caption{$k\_clusters = 5$, $mode = 'random'$}
    \end{minipage}\hfill
    \begin{minipage}{0.45\textwidth}
        \centering
        \includegraphics[scale=0.4]{lena/7_random.png}
        \caption{$k\_clusters = 7$, $mode = 'random'$}
    \end{minipage}
\end{figure}
\begin{table}[htbp]
    \centering
    \begin{tabular}{|c|c|}
        \hline
        \textbf{number of clusters} & \textbf{Total runtime} \\
        \hline
        $3$ & 3.887 seconds \\
        \hline
        $5$ & 4.195 seconds \\
        \hline
        $7$ & 7.182 seconds \\
        \hline
    \end{tabular}
    \caption{Random Mode Demo}
    \label{tab:random_mode}
\end{table}
\clearpage 
\paragraph{In\_pixels Mode}
\quad 

\begin{figure}[htbp]
    \centering
    \begin{minipage}{0.45\textwidth}
        \centering
        \includegraphics[scale=0.4]{lena/original.png}
        \caption{Original}
    \end{minipage}\hfill
    \begin{minipage}{0.45\textwidth}
        \centering
        \includegraphics[scale=0.4]{lena/3_ip.png}
        \caption{$k\_clusters = 3$, $mode = 'in\_pixels'$}
    \end{minipage}
    
    \vspace{0.5cm} % Adjust vertical spacing between rows
    
    \begin{minipage}{0.45\textwidth}
        \centering
        \includegraphics[scale=0.4]{lena/5_ip.png}
        \caption{$k\_clusters = 5$, $mode = 'in\_pixels'$}
    \end{minipage}\hfill
    \begin{minipage}{0.45\textwidth}
        \centering
        \includegraphics[scale=0.4]{lena/7_ip.png}
        \caption{$k\_clusters = 7$, $mode = 'in\_pixels'$}
    \end{minipage}
\end{figure}

\begin{table}[htbp]
    \centering
    \begin{tabular}{|c|c|}
        \hline
        \textbf{number of clusters} & \textbf{Total runtime} \\
        \hline
        $ 3$ & 2.807 seconds \\
        \hline
        $5$ & 3.130 seconds \\
        \hline
        $7$ & 3.852 seconds \\
        \hline
    \end{tabular}
    \caption{In\_pixels Mode Demo}
    \label{tab:in_pixels_mode}
\end{table}
\clearpage % Add this to force figures to appear before starting the next section
\subsubsection{Landscape image}
\quad This image has size 20MB
\paragraph{Random Mode}
\quad 
\begin{figure}[htbp]
    \centering
    \begin{minipage}{0.45\textwidth}
        \centering
        \includegraphics[scale=0.4]{landscape/20mb.png}
        \caption{Original}
    \end{minipage}\hfill
    \begin{minipage}{0.45\textwidth}
        \centering
        \includegraphics[scale=0.4]{landscape/3_random.png}
        \caption{$k\_clusters = 3$, $mode = 'random'$}
    \end{minipage}
    
    \vspace{0.5cm} % Adjust vertical spacing between rows
    
    \begin{minipage}{0.45\textwidth}
        \centering
        \includegraphics[scale=0.4]{landscape/5_random.png}
        \caption{$k\_clusters = 5$, $mode = 'random'$}
    \end{minipage}\hfill
    \begin{minipage}{0.45\textwidth}
        \centering
        \includegraphics[scale=0.4]{landscape/7_random.png}
        \caption{$k\_clusters = 7$, $mode = 'random'$}
    \end{minipage}
\end{figure}
\begin{table}[htbp]
    \centering
    \begin{tabular}{|c|c|}
        \hline
        \textbf{number of clusters} & \textbf{Total runtime} \\
        \hline
        $3$ & 3.887 seconds \\
        \hline
        $5$ & 4.195 seconds \\
        \hline
        $7$ & 7.182 seconds \\
        \hline
    \end{tabular}
    \caption{Random Mode Demo}
    \label{tab:random_mode}
\end{table}
\clearpage 
\paragraph{In\_pixels Mode}
\quad 

\begin{figure}[htbp]
    \centering
    \begin{minipage}{0.45\textwidth}
        \centering
        \includegraphics[scale=0.4]{landscape/original.png}
        \caption{Original}
    \end{minipage}\hfill
    \begin{minipage}{0.45\textwidth}
        \centering
        \includegraphics[scale=0.4]{landscape/3_ip.png}
        \caption{$k\_clusters = 3$, $mode = 'in\_pixels'$}
    \end{minipage}
    
    \vspace{0.5cm} % Adjust vertical spacing between rows
    
    \begin{minipage}{0.45\textwidth}
        \centering
        \includegraphics[scale=0.4]{landscape/5_ip.png}
        \caption{$k\_clusters = 5$, $mode = 'in\_pixels'$}
    \end{minipage}\hfill
    \begin{minipage}{0.45\textwidth}
        \centering
        \includegraphics[scale=0.4]{landscape/7_ip.png}
        \caption{$k\_clusters = 7$, $mode = 'in\_pixels'$}
    \end{minipage}
\end{figure}

\begin{table}[htbp]
    \centering
    \begin{tabular}{|c|c|}
        \hline
        \textbf{number of clusters} & \textbf{Total runtime} \\
        \hline
        $ 3$ & 2.807 seconds \\
        \hline
        $5$ & 3.130 seconds \\
        \hline
        $7$ & 3.852 seconds \\
        \hline
    \end{tabular}
    \caption{In\_pixels Mode Demo}
    \label{tab:in_pixels_mode}
\end{table}
\quad Here is the link to get the image \cite{landscapeimage}

\subsection{Evaluation}

\subsubsection{Runtime}
\begin{itemize}
    
     \textbf{Total Runtime:} Similarly, the total runtime for smaller \texttt{k\_clusters} values is significantly faster compared to larger \texttt{k\_clusters} values. This is because fewer clusters require fewer iterations and computations.
\end{itemize}

\subsubsection{File Size}
\begin{itemize}
    \item \textbf{Impact on Runtime (Input Image):} File size directly impacts runtime due to the number of data points (pixels) that need to be processed. Larger image files with higher resolutions or dimensions require more computation time.
\end{itemize}

\subsubsection{Image Quality}
\begin{itemize}
    \item \textbf{Retention of Details:} Larger values of \texttt{k\_clusters} preserve more details from the original images because they can represent a greater variety of colors. This results in higher fidelity to the original image but comes at the cost of increased computational time.
    \item \textbf{Smaller \texttt{k\_clusters}:} While smaller values of \texttt{k\_clusters} may result in some loss of fine details, they can still capture defining contents and major features of the images.
\end{itemize}

\subsubsection{Conclusion}
\begin{itemize}
    \item \textbf{Trade-off:} There is a trade-off between runtime, file size, and image quality when choosing \texttt{k\_clusters} and processing image files. Smaller values of \texttt{k\_clusters} offer faster runtime but may sacrifice some image details, whereas larger values retain more details but require longer computation times, especially with larger image files.
\end{itemize}

\pagebreak