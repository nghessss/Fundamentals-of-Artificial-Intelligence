\documentclass[12pt]{article}
\usepackage{amsmath}
\usepackage{amsfonts}
\usepackage{float}
\usepackage{fancyhdr}
\usepackage{graphicx}
\usepackage[colorlinks=true,linkcolor=blue, citecolor=red]{hyperref}
\usepackage{url}
\usepackage[top=.75in, left=.75in, right=.75in, bottom=1in]{geometry}
\usepackage{multirow}
\usepackage{enumitem}
\usepackage{setspace}
\usepackage{parskip}
\usepackage{tikz}
\usepackage{forest}
\usepackage{titlesec}
\usepackage{caption}
\usepackage{subcaption}

% For algorithm
\usepackage{algorithm}
\usepackage{algpseudocode}

% For table
\usepackage{array}
\usepackage{diagbox}
% ============ CODE ============
\usepackage{listings}
\usepackage{xcolor}
\definecolor{codegreen}{rgb}{0,0.6,0}
\definecolor{codegray}{rgb}{0.5,0.5,0.5}
\definecolor{codepurple}{rgb}{0.58,0,0.82}
\definecolor{backcolour}{rgb}{0.95,0.95,0.92}

\usepackage[backend=biber, sorting=none]{biblatex}
\addbibresource{ref.bib}

% Styling for the code.
\lstdefinestyle{mystyle}{
    backgroundcolor=\color{backcolour},   
    commentstyle=\color{codegreen},
    keywordstyle=\color{magenta},
    numberstyle=\tiny\color{codegray},
    stringstyle=\color{codepurple},
    basicstyle=\ttfamily\footnotesize,
    breakatwhitespace=false,         
    breaklines=true,                 
    captionpos=b,                    
    keepspaces=true,                 
    numbers=left,                    
    numbersep=5pt,                  
    showspaces=false,                
    showstringspaces=false,
    showtabs=false,                  
    tabsize=2
}
\lstset{style=mystyle}

\lstdefinestyle{terminal}{
    basicstyle=\ttfamily\color{black},
    backgroundcolor=\color{gray!10},
    frame=none, % Add top and bottom frames
    framerule=1pt, % Set the frame rule thickness
    framexleftmargin=10pt, % Adjust the left margin of the frame
    breaklines=true,
    breakatwhitespace=true,
    numbers=none, % Remove line numbers
    keepspaces=false, % Keep spaces
    tabsize=2,
}

% Disable indentation on new paragraphs
\setlength{\parindent}{0pt}

% Define course name, report name and report title.
\newcommand{\coursename}{Fundamentals of Artificial Intelligence}
\newcommand{\coursetitle}{Fundamentals of Artificial Intelligence}
\newcommand{\reportname}{Search}
\newcommand{\reporttitle}{Search}

% Sort by student number
\newcommand{\studentname}{Vo Thanh Nghia (22127295)}
\newcommand{\teachername}{Nguyen Thi Thu Hang \\Pham Trong Nghia \\Bui Duy Dang}

% ============ HEADER AND FOOTER ============
% Header length
\setlength{\headheight}{29.43912pt}

% Footer page number would be on the lower-right corner
\pagestyle{fancy}
\fancyfoot{}
\fancyfoot[R]{\thepage}

\lhead{\reportname}
\rhead{\coursename}

% Set line and paragraph spacing
\renewcommand{\baselinestretch}{1.05}
\setlength{\parskip}{2pt}

% ============ DOCUMENT ============
\begin{document}
\begin{titlepage}
\newcommand{\HRule}{\rule{\linewidth}{0.5mm}}
\centering

\textsc{\LARGE vietnam national university ho chi minh city}\\[0.2cm]
\textsc{\LARGE university of science}\\[0.3cm]
\textsc{\Large faculty of information technology}\\[0.5cm]

\includegraphics[scale=.35]{content/hcmus-logo.png}\\[0.5cm]

% \huge{\bfseries{BÁO CÁO}}\\
\huge{\bfseries{Search}}\\
\textbf{\large COURSE NAME: \coursetitle}\\[0.5cm]

\begin{minipage}[t]{0.45\textwidth}
\begin{flushleft} \large
\emph{Student:}\\
\studentname
\end{flushleft}
\end{minipage}
~
\begin{minipage}[t]{0.45\textwidth}
\begin{flushright} \large
\emph{Lecturer:} \\
\teachername
\end{flushright}
\end{minipage}\\[5cm]

{\large \today}\\[2cm]


\vfill
\end{titlepage}
	
\setcounter{secnumdepth}{4}
\tableofcontents
\pagebreak
\section{Information}
\subsection{Student Information}
\renewcommand{\arraystretch}{2}

\begin{center}
\begin{tabular}{|>{\centering\arraybackslash}m{4cm}|>{\centering\arraybackslash}m{5cm}|>{\centering\arraybackslash}m{7cm}|}
  \hline
  \textbf{\Large Student ID} & \textbf{\Large Full name} & \textbf{\Large Email} \\
  \hline
  \Large 22127295 & \Large Vo Thanh Nghia & \Large vtnghia22@clc.fitus.edu.vn \\
  \hline
\end{tabular}
\end{center}

% \subsection{Source Code}

% \textbf{\href{https://colab.research.google.com/drive/1OryLc0728ZoFUV8M9vRWrRM5Zd4O_uUf?usp=sharing}{Link to Source Code - Google Colab}}


\section{Abstract}

\qquad This report details the implementation and evaluation of five fundamental graph search algorithms: Breadth First Search (BFS), Depth First Search (DFS), Uniform-Cost Search (UCS), Greedy Best First Search (GBFS), and A*. Developed within the provided student\_functions.py framework, the project adheres to specific guidelines and constraints. The report includes a self-assessment of the project's completeness, an exploration of the underlying theories, a comparative analysis of UCS and A* algorithms, and the implementation of additional search algorithms for extra credit. This work demonstrates a comprehensive understanding of various graph search strategies, their theoretical foundations, and practical applications.

\section{source video}
Here is the link to the \href{https://drive.google.com/drive/folders/1f-bkZnT4g-qskH6K7UEXmzV7ogh_MPTy?usp=drive_link}{source video}.
\pagebreak
\section{Output Evaluation}
All the test cases is run with max iterations of 100. Increasing this value may lead to the different output or run time, however, with the small number of clusters, the growth of iteration does not affect the algorithm performance due to the early stopping
\pagebreak
\subsection{Result}
\subsubsection{lena image}
\quad This image's size is 264KB

\paragraph{Random Mode}
\quad 
\begin{figure}[htbp]
    \centering
    \begin{minipage}{0.45\textwidth}
        \centering
        \includegraphics[scale=0.4]{lena/original.png}
        \caption{Original}
    \end{minipage}\hfill
    \begin{minipage}{0.45\textwidth}
        \centering
        \includegraphics[scale=0.4]{lena/3_random.png}
        \caption{$k\_clusters = 3$, $mode = 'random'$}
    \end{minipage}
    
    \vspace{0.5cm} % Adjust vertical spacing between rows
    
    \begin{minipage}{0.45\textwidth}
        \centering
        \includegraphics[scale=0.4]{lena/5_random.png}
        \caption{$k\_clusters = 5$, $mode = 'random'$}
    \end{minipage}\hfill
    \begin{minipage}{0.45\textwidth}
        \centering
        \includegraphics[scale=0.4]{lena/7_random.png}
        \caption{$k\_clusters = 7$, $mode = 'random'$}
    \end{minipage}
\end{figure}
\begin{table}[htbp]
    \centering
    \begin{tabular}{|c|c|}
        \hline
        \textbf{number of clusters} & \textbf{Total runtime} \\
        \hline
        $3$ & 3.887 seconds \\
        \hline
        $5$ & 4.195 seconds \\
        \hline
        $7$ & 7.182 seconds \\
        \hline
    \end{tabular}
    \caption{Random Mode Demo}
    \label{tab:random_mode}
\end{table}
\clearpage 
\paragraph{In\_pixels Mode}
\quad 

\begin{figure}[htbp]
    \centering
    \begin{minipage}{0.45\textwidth}
        \centering
        \includegraphics[scale=0.4]{lena/original.png}
        \caption{Original}
    \end{minipage}\hfill
    \begin{minipage}{0.45\textwidth}
        \centering
        \includegraphics[scale=0.4]{lena/3_ip.png}
        \caption{$k\_clusters = 3$, $mode = 'in\_pixels'$}
    \end{minipage}
    
    \vspace{0.5cm} % Adjust vertical spacing between rows
    
    \begin{minipage}{0.45\textwidth}
        \centering
        \includegraphics[scale=0.4]{lena/5_ip.png}
        \caption{$k\_clusters = 5$, $mode = 'in\_pixels'$}
    \end{minipage}\hfill
    \begin{minipage}{0.45\textwidth}
        \centering
        \includegraphics[scale=0.4]{lena/7_ip.png}
        \caption{$k\_clusters = 7$, $mode = 'in\_pixels'$}
    \end{minipage}
\end{figure}

\begin{table}[htbp]
    \centering
    \begin{tabular}{|c|c|}
        \hline
        \textbf{number of clusters} & \textbf{Total runtime} \\
        \hline
        $ 3$ & 2.807 seconds \\
        \hline
        $5$ & 3.130 seconds \\
        \hline
        $7$ & 3.852 seconds \\
        \hline
    \end{tabular}
    \caption{In\_pixels Mode Demo}
    \label{tab:in_pixels_mode}
\end{table}
\clearpage % Add this to force figures to appear before starting the next section
\subsubsection{Landscape image}
\quad This image has size 20MB
\paragraph{Random Mode}
\quad 
\begin{figure}[htbp]
    \centering
    \begin{minipage}{0.45\textwidth}
        \centering
        \includegraphics[scale=0.4]{landscape/20mb.png}
        \caption{Original}
    \end{minipage}\hfill
    \begin{minipage}{0.45\textwidth}
        \centering
        \includegraphics[scale=0.4]{landscape/3_random.png}
        \caption{$k\_clusters = 3$, $mode = 'random'$}
    \end{minipage}
    
    \vspace{0.5cm} % Adjust vertical spacing between rows
    
    \begin{minipage}{0.45\textwidth}
        \centering
        \includegraphics[scale=0.4]{landscape/5_random.png}
        \caption{$k\_clusters = 5$, $mode = 'random'$}
    \end{minipage}\hfill
    \begin{minipage}{0.45\textwidth}
        \centering
        \includegraphics[scale=0.4]{landscape/7_random.png}
        \caption{$k\_clusters = 7$, $mode = 'random'$}
    \end{minipage}
\end{figure}
\begin{table}[htbp]
    \centering
    \begin{tabular}{|c|c|}
        \hline
        \textbf{number of clusters} & \textbf{Total runtime} \\
        \hline
        $3$ & 3.887 seconds \\
        \hline
        $5$ & 4.195 seconds \\
        \hline
        $7$ & 7.182 seconds \\
        \hline
    \end{tabular}
    \caption{Random Mode Demo}
    \label{tab:random_mode}
\end{table}
\clearpage 
\paragraph{In\_pixels Mode}
\quad 

\begin{figure}[htbp]
    \centering
    \begin{minipage}{0.45\textwidth}
        \centering
        \includegraphics[scale=0.4]{landscape/original.png}
        \caption{Original}
    \end{minipage}\hfill
    \begin{minipage}{0.45\textwidth}
        \centering
        \includegraphics[scale=0.4]{landscape/3_ip.png}
        \caption{$k\_clusters = 3$, $mode = 'in\_pixels'$}
    \end{minipage}
    
    \vspace{0.5cm} % Adjust vertical spacing between rows
    
    \begin{minipage}{0.45\textwidth}
        \centering
        \includegraphics[scale=0.4]{landscape/5_ip.png}
        \caption{$k\_clusters = 5$, $mode = 'in\_pixels'$}
    \end{minipage}\hfill
    \begin{minipage}{0.45\textwidth}
        \centering
        \includegraphics[scale=0.4]{landscape/7_ip.png}
        \caption{$k\_clusters = 7$, $mode = 'in\_pixels'$}
    \end{minipage}
\end{figure}

\begin{table}[htbp]
    \centering
    \begin{tabular}{|c|c|}
        \hline
        \textbf{number of clusters} & \textbf{Total runtime} \\
        \hline
        $ 3$ & 2.807 seconds \\
        \hline
        $5$ & 3.130 seconds \\
        \hline
        $7$ & 3.852 seconds \\
        \hline
    \end{tabular}
    \caption{In\_pixels Mode Demo}
    \label{tab:in_pixels_mode}
\end{table}
\quad Here is the link to get the image \cite{landscapeimage}

\subsection{Evaluation}

\subsubsection{Runtime}
\begin{itemize}
    
     \textbf{Total Runtime:} Similarly, the total runtime for smaller \texttt{k\_clusters} values is significantly faster compared to larger \texttt{k\_clusters} values. This is because fewer clusters require fewer iterations and computations.
\end{itemize}

\subsubsection{File Size}
\begin{itemize}
    \item \textbf{Impact on Runtime (Input Image):} File size directly impacts runtime due to the number of data points (pixels) that need to be processed. Larger image files with higher resolutions or dimensions require more computation time.
\end{itemize}

\subsubsection{Image Quality}
\begin{itemize}
    \item \textbf{Retention of Details:} Larger values of \texttt{k\_clusters} preserve more details from the original images because they can represent a greater variety of colors. This results in higher fidelity to the original image but comes at the cost of increased computational time.
    \item \textbf{Smaller \texttt{k\_clusters}:} While smaller values of \texttt{k\_clusters} may result in some loss of fine details, they can still capture defining contents and major features of the images.
\end{itemize}

\subsubsection{Conclusion}
\begin{itemize}
    \item \textbf{Trade-off:} There is a trade-off between runtime, file size, and image quality when choosing \texttt{k\_clusters} and processing image files. Smaller values of \texttt{k\_clusters} offer faster runtime but may sacrifice some image details, whereas larger values retain more details but require longer computation times, especially with larger image files.
\end{itemize}

\pagebreak
\section{Basic Theories}
\subsection{Breadth First Search (BFS) Algorithm }
\subsubsection*{Concepts}
\begin{itemize}
    \item \textbf{Introduction:} BFS is an algorithm for searching a tree data structure for a node that satisfies a given property. It starts at the tree root and explores all nodes at the present depth prior to moving on to the nodes at the next depth level.
    \item \textbf{Traversal Method:} BFS uses a queue data structure to keep track of nodes to be explored. It starts from a source node and explores all its neighboring nodes level by level.
\end{itemize}

\subsubsection*{Pseudo Code}
\begin{verbatim}
    function BREADTH-FIRST-SEARCH(problem) returns a solution node or failure
        node←NODE(problem.INITIAL)
        if problem.IS-GOAL(node.STATE) then return node
        frontier←a FIFO queue, with node as an element
        reached← {problem.INITIAL}
        while not IS-EMPTY(frontier) do
            node←POP(frontier)
            for each child in EXPAND(problem, node) do
            s←child.STATE
            if problem.IS-GOAL(s) then return child
            if s is not in reached then
                add s to reached
                add child to frontier
    return failure
\end{verbatim}

\subsubsection*{Complexity}
\begin{itemize}
    \item \textbf{Time Complexity:} \(O(V + E)\), where \(V\) is the number of vertices and \(E\) is the number of edges. This is because each vertex and each edge is processed once.
    \item \textbf{Space Complexity:} \(O(V)\), as additional space is required for the queue and the visited set.
\end{itemize}

\subsubsection*{Properties}
\begin{itemize}
    \item \textbf{Completeness:} BFS is complete, meaning it will always find a solution if one exists, given that the graph is finite.
    \item \textbf{Optimality:} BFS is optimal if all edges have the same weight or no weights. It finds the shortest path in an unweighted graph.
    \item \item \textbf{Traversal Type:} UCS explores nodes with the lowest path cost first.
\end{itemize}
\subsection{Depth First Search (DFS) Algorithm }
\subsubsection*{Concepts}
\begin{itemize}
    \item \textbf{Introduction:} Depth First Search (DFS) is a graph traversal algorithm that explores as far as possible along each branch before backtracking. It traverses deeper into the graph, exploring nodes in depth-first order.
    \item \textbf{Traversal Method:} DFS uses a recursion (or stack) to keep track of nodes to be explored. It starts from a source node and explores as far as possible along each branch before backtracking.
\end{itemize}

\subsubsection*{Pseudo Code}
\begin{verbatim}
function RECURSIVE-DFS(node, problem) returns a solution node or failure
    if problem.IS-GOAL(node.STATE) then return node
    
    for each child in EXPAND(problem, node) do
        s ← child.STATE
        if s is not visited then
            mark s as visited
            result ← RECURSIVE-DFS(child, problem)
            if result != failure then return result
    
    return failure
           
function DEPTH-FIRST-SEARCH(problem) returns a solution node or failure
    node ← NODE(problem.INITIAL)
    mark node.STATE as visited
    return RECURSIVE-DFS(node, problem)
\end{verbatim}

\subsubsection*{Complexity}
\begin{itemize}
    \item \textbf{Time Complexity:} \(O(V + E)\), where \(V\) is the number of vertices and \(E\) is the number of edges. In the worst case, DFS explores all vertices and edges of the graph.
    \item \textbf{Space Complexity:} \(O(V)\), as additional space is required for the stack and the visited set.
\end{itemize}

\subsubsection*{Properties}
\begin{itemize}
    \item \textbf{Completeness:} DFS is complete if the search tree is finite, meaning for a given finite search tree, DFS will come up with a solution if it exists. 
    \item \textbf{Optimality:} DFS is not optimal, meaning the number of steps in reaching the solution, or the cost spent in reaching it is high. 
    \item \textbf{Traversal Type:} Recursive DFS is a depth-first traversal.
\end{itemize}

\subsection{Uniform-Cost Search (UCS)}
\subsubsection*{Concepts}
\begin{itemize}
    \item \textbf{Introduction:} Uniform-Cost Search (UCS) is a graph search algorithm that finds the lowest cost path from a starting node to a goal node in a weighted graph.
    \item \textbf{Traversal Method:} UCS uses a priority queue ordered by path cost to expand nodes with the lowest cumulative cost.
\end{itemize}

\subsubsection*{Pseudo Code}
\label{UCS_best_first_search}
\begin{verbatim}
function BEST-FIRST-SEARCH(problem,f) returns a solution node or failure
    node←NODE(STATE=problem.INITIAL)
    frontier←a priority queue ordered by f , with node as an element
    reached←a lookup table, with one entry with key problem.INITIAL and value node
    while not IS-EMPTY(frontier) do
        node←POP(frontier)
        if problem.IS-GOAL(node.STATE) then return node
        for each child in EXPAND(problem, node) do
            s←child.STATE
            if s is not in reached or child.PATH-COST < reached[s].PATH-COST then
                reached[s]←child
                add child to frontier
    return failure
function EXPAND(problem, node) yields nodes
    s←node.STATE
    for each action in problem.ACTIONS(s) do
        s' ←problem.RESULT(s, action)
        cost←node.PATH-COST + problem.ACTION-COST(s, action,s')
        yield NODE(STATE=s', PARENT=node, ACTION=action, PATH-COST=cost)
\end{verbatim}
\begin{verbatim}

function UNIFORM-COST-SEARCH(problem) returns a solution node, or failure
    return BEST-FIRST-SEARCH(problem, PATH-COST)    
\end{verbatim}

\subsubsection*{Complexity}
\begin{itemize}
    \item \textbf{Time Complexity:} \(O((V + E) \log V)\), where \(V\) is the number of vertices and \(E\) is the number of edges. UCS uses a priority queue, and each edge and vertex may be processed multiple times.
    \item \textbf{Space Complexity:} \(O(V)\), as additional space is required for the priority queue and the visited set.
\end{itemize}

\subsubsection*{Properties}
\begin{itemize}
    \item \textbf{Completeness:} UCS is complete, as it will always find a solution if one exists, given non-negative edge costs.
    \item \textbf{Optimality:} UCS is optimal; it finds the lowest cost path if all edge costs are non-negative.
    \item \textbf{Traversal Type:} UCS explores nodes with the lowest path cost first.
\end{itemize}
\subsection{Greedy Best First Search (GBFS) Algorithm }
\subsubsection*{Concepts}
\begin{itemize}
    \item \textbf{Introduction:} Greedy Best First Search (GBFS) is a graph search algorithm that expands the most promising node chosen according to a heuristic function. It does not guarantee optimal solutions but is often efficient in practice for certain types of problems.
    \item \textbf{Traversal Method:} GBFS uses a priority queue ordered by the heuristic value to expand nodes with the most promising estimated cost.
\end{itemize}
\subsubsection{Heuristic function note}

Due to the requirements of this project, the lecturer specified that the heuristic function should be the edge weight, which initially caused some confusion for me.

\begin{figure}[h]
\centering
\includegraphics[]{Greedy_requirement.PNG}
\label{fig:Greedy_requirement}
\end{figure}

Therefore, I have decided to use the edge weight as the heuristic function, representing the direct distance between the current node and the goal node.

The heuristic function is defined as follows:
\[h[node] = matrix[goal][node]\]


\subsubsection*{Pseudo Code}
function BEST-FIRST-SEARCH(problem, f) was mentioned in Section \ref{UCS_best_first_search}.
\begin{verbatim}
function GREEDY-BEST-FIRST-SEARCH(problem) returns a solution node or failure
    return BEST-FIRST-SEARCH(problem, HEURISTIC)

\end{verbatim}

\subsubsection*{Complexity}
\begin{itemize}
    \item \textbf{Time Complexity:} The worst-case time complexity of GBFS is \( O(|V|) \), where \( V \) is the number of vertices. With a good heuristic function, however, the complexity can be substantially reduced, potentially reaching \( O(b^m) \) on certain problems, where \( b \) is the branching factor and \( m \) is the maximum depth of the search.
    \item \textbf{Space Complexity:} GBFS has a space complexity of \( O(|V|) \), primarily due to the priority queue and the visited set.
\end{itemize}

\subsubsection*{Properties}
\begin{itemize}
    \item \textbf{Completeness:} GBFS is complete in finite state spaces but may not be complete in infinite state spaces due to the potential for encountering loops or cycles.
    \item \textbf{Optimality:} GBFS is not generally optimal because it does not consider the total path cost but relies solely on the heuristic function to guide the search.
    \item \textbf{Traversal Type:} GBFS explores nodes based on the heuristic value, prioritizing nodes that appear to be closest to the goal according to the heuristic.
\end{itemize}
\subsection{A* Search Algorithm}
\subsubsection*{Concepts}
\begin{itemize}
    \item \textbf{Introduction:} A* Search is a graph search algorithm that combines the advantages of both Dijkstra's algorithm and greedy best-first search. It uses both the actual cost from the start node (g-value) and an estimated cost to the goal node (h-value) to guide the search.
    \item \textbf{Traversal Method:} A* Search uses a priority queue ordered by the sum of g-value and h-value (f-value) to expand nodes with the lowest estimated total cost first.
\end{itemize}
\subsubsection{Heuristic function note}
I code 2 version of Astar algorithm, one with the heuristic function as the Euclidean distance, and the other with the heuristic function as the Manhattan distance. 

The Euclidean distance is calculated as follows:
\[h(node) = \sqrt{(goal_x - node_x)^2 + (goal_y - node_y)^2}\]
The Manhattan distance is calculated as follows:
\[h(node) = |goal_x - node_x| + |goal_y - node_y|\]
\subsubsection*{Pseudo Code}
function BEST-FIRST-SEARCH(problem, f) was mentioned in Section \ref{UCS_best_first_search}.
\begin{verbatim}
function A-STAR-SEARCH(problem) returns a solution node or failure
    return BEST-FIRST-SEARCH(problem, g+h)
\end{verbatim}

\subsubsection*{Complexity}
\begin{itemize}
    \item \textbf{Time Complexity:} In the worst case, A* Search has a time complexity of \( O(b^d) \), where \( b \) is the branching factor of the graph and \( d \) is the depth of the optimal solution. However, with a good heuristic function, it often performs much better.
    \item \textbf{Space Complexity:} A* Search has a space complexity of \( O(|V|) \), where \( V \) is the number of vertices, due to the priority queue and the visited set.
\end{itemize}

\subsubsection*{Properties}
\begin{itemize}
    \item \textbf{Completeness:} A* Search is complete if a solution exists, given that the branching factor is finite and all edge costs are non-negative.
    \item \textbf{Optimality:} A* Search is optimal if the heuristic function \( h(s) \) is admissible (never overestimates the true cost to reach the goal) and consistent (satisfies the triangle inequality).
    \item \textbf{Traversal Type:} A* Search explores nodes based on the f-value, which combines the g-value (cost from start to current node) and the h-value (estimated cost from current node to goal).
\end{itemize}
\subsection{Iterative Deepening Search (IDS) Algorithm}
\subsubsection*{Concepts}
\begin{itemize}
    \item \textbf{Introduction:} Iterative Deepening Search (IDS) combines the benefits of depth-first and breadth-first search. It performs a series of depth-limited searches, increasing the depth limit with each iteration.
    \item \textbf{Traversal Method:} IDS uses a depth-limited search repeatedly, increasing the limit after each complete pass until a solution is found or the search space is exhausted.
\end{itemize}

\subsubsection*{Pseudo Code}
\begin{verbatim}
function ITERATIVE-DEEPENING-SEARCH(problem) returns a solution node or failure
    for depth = 0 to oo do
        result←DEPTH-LIMITED-SEARCH(problem, depth)
        if result != cutoff then return result

function DEPTH-LIMITED-SEARCH(problem, l) returns a node or failure or cutoff
    frontier←a LIFO queue (stack) with NODE(problem.INITIAL) as an element
    result←failure
    while not IS-EMPTY(frontier) do
        node←POP(frontier)
        if problem.IS-GOAL(node.STATE) then return node
        if DEPTH(node) > l then
            result←cutoff
        else if not IS-CYCLE(node) do
            for each child in EXPAND(problem, node) do
                add child to frontier
    return result
\end{verbatim}

\subsubsection*{Complexity}
\begin{itemize}
    \item \textbf{Time Complexity:} \(O(b^d)\), where \(b\) is the branching factor and \(d\) is the depth of the shallowest goal node. This is because IDS performs multiple passes, but the time complexity remains exponential.
    \item \textbf{Space Complexity:} \(O(bd)\), as IDS only needs to store a stack of nodes up to the current depth limit.
\end{itemize}

\subsubsection*{Properties}
\begin{itemize}
    \item \textbf{Completeness:} IDS is complete if the branching factor is finite, meaning it will eventually find a solution if one exists.
    \item \textbf{Optimality:} IDS is optimal if all step costs are equal, as it finds the shallowest goal node.
    \item \textbf{Traversal Type:} IDS performs iterative deepening, combining depth-first search's space efficiency with breadth-first search's completeness.
\end{itemize}

\pagebreak
\section{Comparison of UCS and A* Algorithms}

\begin{table}[h!]
    \centering
    \begin{tabular}{|l|p{6cm}|p{6cm}|}
        \hline
        & \textbf{Uniform-Cost Search (UCS)} & \textbf{A* Search} \\ \hline
        \textbf{Heuristic Function} & 
        Does not use a heuristic function. It purely considers the path cost from the start node to the current node ($g(n)$). & 
        Incorporates a heuristic function ($h(n)$) in addition to the path cost. The total cost function for A* is $f(n) = g(n) + h(n)$, where $h(n)$ is an estimate of the cost from the current node to the goal. \\ \hline
        
        \textbf{Cost Evaluation} & 
        Chooses the node with the minimum path cost ($g(n)$) among nodes that have not been visited but have a neighbor that has been visited. & 
        Chooses the node with the minimum combined cost ($g(n) + h(n)$) among nodes in the frontier. \\ \hline
        
        \textbf{Optimality} & 
        \multicolumn{2}{|p{12cm}|}{Guaranteed to find an optimal solution if the heuristic used in A* is admissible (never overestimates the true cost).} \\ \hline
        
        \textbf{Time and Space Complexity} & 
        \multicolumn{2}{|p{12cm}|}{Have the same time and space complexity in the worst case. However, A* can be more efficient in practice due to the heuristic guiding the search more effectively.} \\ \hline
        
        \textbf{Completeness} & 
        \multicolumn{2}{|p{12cm}|}{Complete, meaning if there is a solution, both algorithms will find it.} \\ \hline
        
        \textbf{Practical Benefit} & 
        Does not benefit from heuristic guidance and may explore more nodes than necessary. & 
        Allows the use of a heuristic, which in non-worst-case scenarios can significantly reduce the number of nodes explored, providing a large practical benefit. \\ \hline
    \end{tabular}
    \caption{Comparison of UCS and A* Algorithms}
    \label{tab:ucs_vs_astar}
\end{table}

\section{Reference}
\defbibheading{bibliography}[\refname]{}
\renewcommand*{\bibfont}{\large}
\nocite{*}  
\printbibliography
\section{Collaborators}
\begin{itemize}
    \item \href{https://chatgpt.com/}{CHATGPT}
    \item \href{https://github.com/features/copilot}{COPILOT}
\end{itemize}
\end{document}